\documentclass{article}
\usepackage{graphicx}
\usepackage{float}
\usepackage{booktabs}
\usepackage[colorlinks=true, urlcolor=blue, linkcolor=red]{hyperref}
\begin{document}
	
	\title{Decision Tree Algorithm Assignment}
	\author{K.N.Toosi University of Technology\\Introduction to Data Mining}
	\date{Fall 2024}
	\maketitle
	\newpage
	\part{Practical Assignment}
	\section*{Dataset}
	\begin{table}[h]
	\centering
	\begin{tabular}{cccc}
		\toprule
		Age & Income & Married & Buys \\
		\midrule
		22 & High & No & No \\
		35 & Low & Yes & Yes \\
		25 & Medium & No & Yes \\
		45 & Medium & Yes & Yes \\
		50 & High & Yes & Yes \\
		30 & Low & No & No \\
		40 & High & No & Yes \\
		20 & Low & No & No \\
		50 & Low & Yes & Yes \\
		35 & Medium & No & Yes \\
		\bottomrule
	\end{tabular}
	\caption{binary customer data}
	\end{table}
	\section*{Task}
	Create a decision tree using the dataset above. Determine the best criteria for splitting the data at each node and illustrate the resulting tree. Discuss the process you followed and the reasoning behind your decisions.
	\part{Practical Assignment}
	\section*{Dataset}
	\begin{table}[h]
	\centering
	\begin{tabular}{cccc}
		\toprule
		Bedrooms & Square\_Feet & House\_Age & Price  \\
		\midrule
		3        & 1500        & 10        & 300000 \\
		4        & 2000        & 5         & 450000 \\
		2        & 900         & 30        & 150000 \\
		3        & 1800        & 20        & 350000 \\
		5        & 2500        & 8         & 550000 \\
		4        & 2200        & 15        & 500000 \\
		2        & 1200        & 25        & 200000 \\
		3        & 1600        & 10        & 310000 \\
		4        & 2100        & 7         & 480000 \\
		5        & 3000        & 3         & 600000 \\
		\bottomrule
	\end{tabular}
	\caption{house pricing data}
	\end{table}
	\section*{Task}
	Create a regression tree using the dataset above. Determine the best criteria for splitting the data at each node and illustrate the resulting tree. Discuss the process you followed and the reasoning behind your decisions.
	\section*{Performance Metric}
	For regression trees, use the Mean Squared Error (MSE) to measure performance. The MSE is calculated as follows:
	
	\[
	MSE = \frac{1}{n} \sum_{i=1}^{n} (y_i - \hat{y}_i)^2
	\]
	
	where \( y_i \) is the actual value and \( \hat{y}_i \) is the predicted value.
	
	\part{Implementation Assignment}
	\section*{Dataset}
	The dataset for this assignment is the COVID-19 dataset available on Kaggle: \href{https://www.kaggle.com/datasets/meirnizri/covid19-dataset}{COVID-19 Dataset}.
	
	\section*{Task}
	Implement a decision tree classification algorithm in Python and compare it to the Sklearn implementation. Unlike Sklearn, do not use one-hot encoding; manually split multi-label categorical features. Compare your performance with the Sklearn decision tree. Performance should be measured via F1-score.
	\section*{F1-score}
	The F1-score is the harmonic mean of precision and recall. It is calculated as follows:
	$$
	F1 = 2 \cdot \frac{{Precision \cdot Recall}}{{Precision + Recall}}
	$$
	where:
	
	$$
	Precision = \frac{{True Positives}}{{True Positives + False Positives}}
	$$
	
	$$
	Recall = \frac{{True Positives}}{{True Positives + False Negatives}}
	$$
	
	\section*{Note}
	Any attempt to use AI tools for generating the code is strictly prohibited. Students will be asked to present and explain their code during a class session.
\end{document}
